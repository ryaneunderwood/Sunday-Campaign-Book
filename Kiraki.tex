\chapter{Kiraki}\label{Kiraki}
\section{The Continent of Sunday}
\DndDropCapLine{T}{he countries of the continent of} Sunday have, for the most part, agreed to follow similar rules to keep track of time. The world Sunday inhabits is similar to Earth in that it orbits a star in about 360 days, and the planet is inclined on an axis similar to Earth's, so the seasons function similarly. The continent of Sunday is entirely in the southern hemisphere or the planet, leading to an cold polar region at the southern end of the continent, with more tropical regions along the northern end. There is a single large moon, not quite double the size of Earth's, which can cause drastic tides. 
\subsection{Calendar}
The year is broken into 12~months, each of which are 30~days. Months are divided into five~weeks of six~days. The months are named similarly across the continent, in honor of the gods, as follows:
\subparagraph{Ashinry} The first month of the year, this month starts on the day after the shortest day of the year. The first day of the year is called Annulus Crossing, and is a holiday celebrated continent wide.
\subparagraph{Daleduary} The second month of the year is very cold across most of the continent south of Sigasade, Tadau Koturu, and Nedelja.
\subparagraph{Yarfinnus} Another cold month, though spring approaches along the temperate east coast.
\subparagraph{Merorise} The spring Equinox marks the first day of Merorise. In most of Kiraki, spring comes in earnest towards the end of the month.
\subparagraph{Harpalakus} A windy month in the interior of Kiraki. 
\subparagraph{Taranius} The summer solstice and longest day of the year is the last day of Taranius, a holiday celebrated continent-wide.
\subparagraph{Praximas} 
The seventh month of the year. Compact day, commemorating the signing of the Compact of the 7 Races and establishing the Commonwealth of Kiraki is celebrated on the 15th.
\subparagraph{Ionis} The hottest month of the year in Kiraki.
\subparagraph{Taygana} The summer winds down. \textbf{Seafolk} and \textbf{kobolds} celebrate Pretaslava, during the last full moon of this month. The festival is in worship of the great dragon Preta, who claims to be an avatar of Taygeta and Pasiphae.
\subparagraph{Tesuncta} The first day of this month marks the autumn equinox. Various harvest festivals across the continent take place during this month.
\subparagraph{Themistus} The autumn days begin to grow shorter and colder.
\subparagraph{Meneasmas} By the end of the month, most of Kiraki save for the eastern coast, has usually experienced at least one snow. Across the continent, various holidays mark the end of the year. The last day is the shortest day of the year, which is why it is named in honor of Meneas, god of shadows.
\subsection{Days of the Week}
There are six days of the week. Like the months, they also honor the gods in name:
\subparagraph{Amalday} The first day of the week. 
\subparagraph{Khofday} Goblins honor their creator and abstain from breakfast on Khofdays.
\subparagraph{Lysiday} In Kiraki, The third Lysiday of each month is called Middle Day, and is a day of rest for most.
\subparagraph{Tsaday} Thirsty Tsaday's might have drink specials on at the pub.
\subparagraph{Sinoday} Gladiatorial games or combats of any sort are usually held Sinoday evenings.
\subparagraph{Pasiday} The last day of the week. In many cultures, a day of rest. Gnomes ignore the weekly rest day, as do seafolk fishermen.
\subsection{The Year}\label{Year} The main campaign should take place in the year 401 PCS (Post- Compact of the Seven). The various kingdoms and citystates of the region all used different calendars before koining together as Kiraki. Other nations, such as Amcys, use a different year 0. In the Amcysian calendar, they are in year 2801 AG (Amcysia Gloria), following the city-state of Amcysia conquering the rest of the country.

\section{Factions}
The continent of Sunday is full of people with competing interests. Some vie for personal power, some fight for glory, others do their best to protect their people or compatriots. This section will cover the major factions and characters that have recurring appearances throughout the story, and play a part in shaping the world. Smaller factions and minor characters which only hold local sway will be introduced in the chapter pertaining to that part of the adventure. The Dungeon Master should be familiar with the characters and factions in this section before the adventure begins, because many of these characters could or should be known by the players just by virtue of living in Kiraki.
\subsection{The Council of Seven}
The ruling body of Kiraki is a council made of seven members, one from each of the representative races of Kiraki: humans, seafolk, goblins, arboreans, gnomes, elves, and dwarves. The council of seven has been in place for four hundred years, and makes the laws and governs the commonwealth.\\
Each of the members of the council has an additional role in governing. One member is the chief general of the Kiraki Army, one member is the chief general of the Kiraki Navy, one member is the treasurer, one member is the agriculture chief, one member is the infrastructure chief, one member is foreign relations liaison, and one member is the magic expert.\\
Each of the races selects their representative in different ways. The dwarves and elves pass the leadership on through family chains, with the dwarves having a king, and the elves an emperor. Seafolk are a matriarchal society, with the most powerful and admired woman naturally emerging as a leader the other seafolk look to. Gnomes elect a mayor of Silver Town, who serve as their representative, and humans do similarly for the mayor of Gold City. Goblin society can be fairly cutthroat, and there can be challenges of combat to become the tribe chieftain. Arboreans usually care little for leadership. One particular clan of hardwood arboreans, the family of the Stoutbarrels, assumed leadership of the Kiraki Army, and no other arborean has ever expressed interest in taking over.
\subsubsection{Fenjassa Kothelja}
Fenjassa (LG female seafolk storm herald barbarian) is a council member from Obron village, representing her seafolk as the chief general of the Kiraki navy. She is 48 years old. She has grey-blue skin, braided blue dreadlocks, large muscles, and wields a Thunder Axe (see Chapter~\ref{Items}). She is the third longest tenured on the council, having served on it for 20 years.\\
Fenjassa is ever aware of the attempts of Arifjire Kricacniac to usurp her. She tolerates Ari, preferring to win over her people's loyalty directly, and refuses to step into the mud and play Ari at her own games.
\subparagraph{Personality} Fenjassa is stern and gruff, but beneath her outer personality she is kind and loyal. She is wary of outside interence in Obron Village, but always fair in her treatment of all.
\subparagraph{Ideal} Fenjassa believes in protecting the seafolk. She worries that as the smallest population group in Kiraki, they need protection from the other races.
\subparagraph{Bond} Jarvull Doraslav, the retired navyman is a long time on and off love interest of Fenjassa's. Her loyalties lie first with her people though, which has prevented them from becoming a long term couple.
\subparagraph{Flaw} Fenjassa is stubborn, and unwilling to counter the games that Ari sets in motion, which puts her position in jeopardy.

\subsubsection{Dwovil Durnstutter}
Dwovil (LG male dwarf oath of glory paladin) is King under the Mountain in Mt. Silicon, and the foreign relations liaison on the council of seven. He is 
\subparagraph{Personality}
\subparagraph{Ideal}
\subparagraph{Bond}
\subparagraph{Flaw}

\subsubsection{Tiberius Stoutbarrel}
Tiberius (LN male arborean champion fighter)
\subparagraph{Personality}
\subparagraph{Ideal}
\subparagraph{Bond}
\subparagraph{Flaw}

\subsubsection{Shakarr}
Shakarr (CN male human chronurgy wizard)
\subparagraph{Personality}
\subparagraph{Ideal}
\subparagraph{Bond}
\subparagraph{Flaw}

\subsubsection{Creedent the Fickle}
Creedent (TN female goblin observer of mind mentalist)
\subparagraph{Personality}
\subparagraph{Ideal}
\subparagraph{Bond}
\subparagraph{Flaw}

\subsubsection{Riboton Biboton}
Riboton (NE male Gnome illusion wizard) is the mayor of Silver Town and the treasurer on the council of seven. He is of older middle age for a gnome, with crooked glasses and short tousled brown hair. He was voted in as mayor after the unfortunate murder of the previous mayor and his best friend Arghin Fradicius --- a murder he orchestrated. \\
Riboton is secretly a member of the Undercouncil, a shadowy group that is unknown to most of the official Council of Seven members. He is the newest member of the Council, after only having been elected one year prior.
\subparagraph{Personality}
Beneath his kindly exterior and jovial gnomish nature lies a ruthless cutthroat, willing to kill anyone to obtain or retain power. He acts kindly and with a nervous tick, but it is all a facade meant to deceive people into trusting him.
\subparagraph{Ideal}
Steer Kiraki in a direction favorable to his interests.
\subparagraph{Bond}
Lucius Fradicius, son of Arghin. After murdering his father, Riboton adopted Lucius, and raised him as his own son. He loves Lucius, but not enough to give up any of his own power.
\subparagraph{Flaw}
In order to gain power, he has allied with some unstable and unreliable people such as Rokas.

\subsubsection{Gaelin Nightstone}
Gaelin (CE male wood elf necromancer)
\subparagraph{Personality}
\subparagraph{Ideal}
\subparagraph{Bond}
\subparagraph{Flaw}
\subsubsection{Historical Council of 7}
Kipin Fradicius, Talus Cherry, Xanila Flightfoot, Brutus Stoutbarrel, Coughlin MacCreedy, Xyfan the Vindictive, Minja Ricic, Malvol Lumena, Quegwa the Wise
\subsection{The Undercouncil}
\subsection{The Anti-Chaos Taskforce (ACT)}
Jes Horatio, Agent E,F,H,I, Laila/L
\subsection{The Seekers}
Jarvull Doraslav, 
\subsection{The Department of Manipulations and Plans (DMP)}
Riboton, Mayor Pythagora, Rokas
\subsection{The Cult of the Sun}
\subsubsection{Ilmaliar Lumena}
Ilmaliar (LE female tiefling noble)
\subparagraph{Personality}
\subparagraph{Ideal}
\subparagraph{Bond}
\subparagraph{Flaw}
\subsubsection{Mercury Shumaker}
Mercury (LE male human cultist)
\subparagraph{Personality}
\subparagraph{Ideal}
\subparagraph{Bond}
\subparagraph{Flaw}

\subsection{Preta's Flock}
\subsubsection{Arifjire Kricacniac}
Ari (LEG female seafolk sorceress)
\subparagraph{Personality}
\subparagraph{Ideal}
\subparagraph{Bond}
\subparagraph{Flaw}
\subsection{The Pirates of Piva Pava}
\subsubsection{Grand Jean Argenterie}
Grand Jean (CE male kobold sailor)
\subparagraph{Personality}
\subparagraph{Ideal}
\subparagraph{Bond}
\subparagraph{Flaw}
\subsection{Humans of Kiraki}
Ignotius Cherry, Maple Inverness, Shakarr, Obi Steele
\subsection{Arboreans of Kiraki}
Palma (Obron), kastanja (Belondir woods), Tiberius, 
\subsection{The Gnomes of Silver Town}
Riboton, Buzeldorf, Lucius, Kramic, Stingy
\subsection{The Dwarves of Mt. Silicon}
Dwovil, Dwolin, Trellor Ormund
\subsection{The Seafolk of Obron Village}
Ari, Fenjassa, Jarvull
\subsection{The Elves of Belondir}
Gaelin, Isla
\subsection{The Goblins of Vandalgrim}
Creedent, Grishky
\subsection{The Orcs of Sinopus}
Tonis, Maven, Rashok, Darkun, Togrid, Grik, Bashaar, Maelle the Valiant
\subsection{The Tieflings of Itvar}
Sacrave Lumena, Malvol XIV, Ilmaliar, Vacriss Nodelia
\subsection{Amcysians}
Emperor Tin Zu, Agent 4726, Tugbot, Makusan Ringsunner
\subsection{The Candlestick Order}
\subsection{The Church of Darkness}
