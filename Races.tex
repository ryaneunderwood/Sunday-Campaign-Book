\chapter{New Playable Races}\label{Races}
\section{Arborean}
\DndDropCapLine{A}{wondrous species, isolated but loyal.} \textit{Though their race lives in the woods not but a few miles from my home, I rarely get chance to see one up close. Outside of us tieflings, arboreans seem to be the most saddened by the treatment of the orcs. I hope to forge an alliance with them. I shall seek them out east of Belondir.}\\
 \begin{flushright}---Malmalor Lumena \end{flushright}
The woods are a place where every sentient being knows: \textit{I am not alone}. There are animals of all scales, some predators, and some prey. A traveler's eyes and ears will scan for dangers, for in the forest there are many. Dark magics can collect in forgotten forests, and so arboreans make sure the forests are not forgotten.\\
Arboreans are tree-like beings, with rough bark for skin.
\subsection{Merope's Chosen}
Merope, inspired by the bountiful green life of the material plane drew heavy inspiration from trees when she developed her Chosen Race. The goddess of plants and nature drew heavy inspiration from trees when crafting Oaknaught, the first arborean. Seeing the efficient design of the bipedal forms her fellow gods had created, Merope followed suit, but made sure to preserve that which inspired her about trees: their hardy bark, their greenery, and most of all their peaceful nature.\\
Like the trees they were inspired from, Merope chose to make Arboreans varied, adapted to the specific environment of their creation. In the hardy deciduous forests of southern Kiraki, hardwood arboreans patrol the Belondir Glades, keeping them clear of underbrush with plenty of space for sunlight to break through. In the orchards and farmlands of central Kiraki, fruiting arboreans are plentiful and cheery. In the frigid snowy mountains where only conifers thrive, evergreen arboreans ensure that their forests do not fall prey to avalanches. In the lush jungles north of Obron Village, and in Tadau Kotoru, Palmetto arboreans sway in the breeze and enjoy the sun. Petrified wood arboreans aimlessly roam near the bogs around Barium Cemetery, a spark of (un)life that Merope did not intend.
\subsection{Shepherds of the Forest}

\subsection{Arborean Traits}
Your Arborean Character has a variety of natural properties and abilities, owing to its treelike properties and close connection to nature.
\subparagraph{Ability Score Increase}Your Constitution score increases by 2.
\subparagraph{Age}Arboreans reach physical maturity around 25, then typically live up to 300 years old.
\subparagraph{Alignment}Arboreans tend to neutral good alignments, though if their forests have been damaged they can venture into chaotic evil territory as they seek vengeance upon those who hurt their forests.
\subparagraph{Size}Arboreans tend to be between 7 and 8 feet tall, and weigh between 400 and 800 pounds. Your size is Medium.
\subparagraph{Creature Type}You count as both a humanoid and plant creature.
\subparagraph{Speed}Your base walking speed is 25 feet. 
\subparagraph{Photosynthesis}As long as you receive four hours of sunlight in a day, you do not need to eat. Going without sunlight for longer than a week will give a level of exhaustion. An additional point of exhaustion is gained for each week spent without sunlight.
\subparagraph{Rootsense}You can sense the location of any creature making contact with nonmagical unworked earth within 30 feet.
\subparagraph{Bark}Your AC increases by 1, and you have resistance to non-magical bludgeoning damage. You have susceptibility to fire damage, and spells or attacks involving fire damage ignore this bonus to AC. You can still gain the normal benefits of a shield if otherwise allowed.
\subparagraph{Languages}You can speak and read Common, Druidic, and Arborean. Arborean is a low guttural language that takes many minutes to communicate what other languages could communicate in a short sentence.
\subparagraph{Subrace}
There are several subraces of arboreans. Choose one.
\begin{DndSidebar}[float=!b]{Peoples of the Woods}
  Arboreans are one of the seven races represented by Kiraki's Council of Seven. They are the pet creation of Merope, goddess of plants, and they act accordingly. Arboreans dedicate their lives to preserving forests. Of all of the races created by the gods of Sunday, arboreans are probably the most welcomed anywhere they are found.
\end{DndSidebar}
\subsubsection{Hardwood Arborean}
Hardwood Arboreans live in the southern forests of Kiraki, as well as in other hardy deciduous forests throughout the continent of Sunday. They are in tune with the seasonal changes of the world, and can present in four different forms depending on which seasonal affect feels appropriate to them. Hardwood arboreans have extremely protective instincts over their home forest, and will go to extreme lengths to protect the ecosystems there.
\subparagraph{Ability Score Increase}Your Strength score increases by 1.
\subparagraph{Extra Sturdy}When you are not wearing armour, your base AC increases by an additional 1, for a total of +2, with all of the other rules of Bark applied to this bonus.
\subparagraph{Seasonal Form}Over the course of a long rest, you can choose whether or not to advance your seasonal appearance to a different one of the four seasonal forms you can take, which each grant a different ability. Your season does not need to match the real world season, but a hardwood arborean usually chooses to present this way unless great need dictates otherwise.
\subparagraph{Spring Form}\textit{New Life:} You can use a bonus action to regain a number of hit points equal to your level. You can do this three times per long rest.
\subparagraph{Summer Form}\textit{Animal Companion:} An animal companion will take roost in your branches. You can cast Find Familiar as an action once per long rest without using a spell slot. The familiar disappears after a number of hours equal to half your level, rounded up.
\subparagraph{Autumn Form}\textit{Vibrant Foliage:} Your entrancing appearance causes you to gain proficiency in Persuasion, Deception, and Performance checks while you are in this seasonal form.
\subparagraph{Winter Form}\textit{Unburdened:} Without leaves to slow you down, your speed increases by 10 ft.

\subsubsection{Evergreen Arborean}
Evergreen arboreans tend to live in the coniferous forests along the mountain ranges bordering Kiraki. Their needly exterior keeps others at bay, which is just the way they like it. They are the most insular type of Arborean, and are rarely found wandering into cities. They can be mistrusting, but are usually fiercly loyal once friendship is established.
\subparagraph{Ability Score Increase}Your Wisdom score increases by 1.
\subparagraph{Spiky}Your needled exterior causes all enemy creatures who make contact with you, or make successful melee attacks against you from within 5 feet of you to make a Dexterity saving through (DC 10+proficiency). On a fail, they take 1d6 piercing damage. A creature can only take damage from this effect once per round.
\subparagraph{Pitch}As an action, you spray the ground within a 30 ft radius circle of you with sticky pitch, making it difficult terrain for non-plant creatures. You can use this feature twice per long rest.

\subsubsection{Palmetto Arborean}
Guardians of the jungles and palm forests near Obron Village in Kiraki, as well as on the Island of Piva Pava, and the nation of Tadau Koturu, Palmetto arboreans are the most outgoing of the arborean subraces. They will leave their forests to party and celebrate, and are known for their elegant dancing.
\subparagraph{Ability Score Increase}Your Dexterity score increases by 1.
\subparagraph{Stormweathered}You are unshakeable, and can choose to be unaffected by any spells or effects involving being moved or damaged by wind. You also have resistance to force damage.
\subparagraph{Coconut Drop}As a reaction to being hit with a melee attack, you can choose to drop a coconut from height on the attacker. The attacker takes 1d8 bludgeoning damage. Additionally, if the creature has a head and is not wearing a helmet, and the creature is of the same size or smaller than you (or its head is below yours), it must make a Constitution saving throw (DC = 8 + your Constitution modifier + your proficiency bonus) or be stunned for one round. You can use this feature twice, and it recharges after a long rest.

\subsubsection{Fruiting Arborean}
Native to the wooded areas near the farmlands in north and central Kiraki, as well as in Sunnudagar and Nedelja, fruiting arboreans are known for their generosity and kind nature. They often shepherd small troops of animals and keep them fed from their fruit. When they wander into civilization, they are greeted with celebration and joy.
\subparagraph{Ability Score Increase}Your Charisma score increases by 1.
\subparagraph{Fruiting Bounty}As long as you have a long-rest over fertile soil, and have taken in at least four hours of sunlight within the last week, you can produce enough fruit to feed one person from each of your four limbs. The type of fruit is set from birth, unless you graft a limb from another fruiting arborean. The fruit eaten will grant the consumer an additional bonus. It takes one minute to eat the fruits, and the advantage the fruit grants lasts two hours.\\
\textit{Peach: }Those who eat your fruits gain advantage on Charisma Saving throws.\\
\textit{Apple: }Those who eat your fruits gain advantage on Intelligence Saving throws.\\
\textit{Pecan: }Those who eat your fruits gain advantage on Constitution Saving throws.\\
\textit{Cherry: }Those who eat your fruits gain advantage on Dexterity Saving throws.\\
\textit{Pear: }Those who eat your fruits gain advantage on Wisdom Saving throws.\\
\textit{Grapefruit: }Those who eat your fruits get advantage on Strength Saving throws.\\
\subparagraph{Grafting}You can gain the fruiting bounties of other Fruiting Arboreans. You can do this by choosing to exchange a limb with them. The fruit off of each limb can feed one person enough to gain the benefit. The limb will take 2d4 days to heal and function properly again. If it is an arm, you will be unable to hold weapons or other items heavier than one pound until it is healed. If it is a leg, your speed is halved and you automatically fail Dexterity based saving throws until it is healed.
\subsubsection{Petrified Wood Arborean}
Deep beneath the muddy bogs to the north of Barium Cemetery, dead trees became petrified, and then were resurrected as undead arboreans. These arboreans aimlessly roam Barium Cemetery looking for purpose, not accepted by most of their kind.
\subparagraph{Ability Score Increase}Your Intelligence Score Increases by 1.
\subparagraph{Languages}You are fluent in Undercommon rather than Druidic.
\subparagraph{Creature Type}Rather than count as a plant creature, you count as undead, as well as humanoid.
\subparagraph{Fossilized}You resist, rather than have susceptibility to fire damage.
\subparagraph{Darkvision}Resurrected in the caves below sunken bogs, rather than \textbf{Rootsense}, you have superior vision in dark and dim conditions. You can see in dim light within 60 feet of you as if it were bright light, and in darkness as if it were dim light. You can’t discern color in darkness, only shades of gray.
\raggedbottom
\pagebreak
\section{Crystallin}
Crystallin are a rare and enigmatic race of humanoid beings with ice-blue skin, giving them an ethereal and otherworldly appearance. Their most distinctive feature is the shards of a shimmering, ice-like material that protrude from their spines, glistening like diamonds in the sunlight. Crystallin are a hardy people, well-suited to the harsh and frigid mountain environments where they make their homes.
\subsection{Carnivorous Hunters}
Crystallin are strict carnivores, feeding exclusively on the meat of mountain creatures that they hunt and trap. They have evolved to survive in areas where plant life is scarce, and their bodies are able to derive all the nutrients they need from the flesh of their prey. This has led them to develop a unique culture around the hunting and preparation of meat, with elaborate rituals and ceremonies surrounding the consumption of each kill.\\
Crystallin are a solitary people, preferring to live in small, tight-knit communities scattered across the mountains. They believe in self-sufficiency, with each individual responsible for their own survival. This has led to a culture of individualism, with each Crystallin honing their skills as hunters, trappers, and craftsmen.\\
Crystallin are skilled in the use of weapons made from the ice-like material that protrudes from their spines. They fashion sharp, deadly blades and spears, as well as powerful shields that can withstand even the strongest blows. They also have a deep understanding of ice, which they use to shape the frozen landscape around them.\\
Crystallin are wary of outsiders, preferring to keep to themselves and protect their way of life. However, those who are able to earn their trust are welcomed with open arms, and the Crystallin are fiercely loyal to their allies. They are a people of great strength and resilience, shaped by the harsh and unforgiving environment in which they live.
\subsection{Crystallin Traits}
Your Crystallin Character has a variety of abilities originating from its connection to Daled, the god of ice.
\subparagraph{Ability Score Increase}Your Constitution score increases by 2 and your Intelligence score increases by 1.
\subparagraph{Age}Crystallin reach physical maturity around 16, then typically live up to 80 years old.
\subparagraph{Alignment}Crystallin often tend towards neutral alignments. They tend to value their clan first, but are rarely outright cruel.
\subparagraph{Size}Crystallin tend to be between 5 and a half and 6 and a half feet tall, and weigh between 180 and 260 pounds. Your size is Medium.
\subparagraph{Speed}Your base walking speed is 30 feet.
\subparagraph{Hunter's Nature} You have advantage on Survival checks and Investigation or Perception checks that involve tracking creatures.
\subparagraph{Blood of Ice} You have resistance to cold damage. Any creature that grapples you or restrains you with its body must make a Constitution saving throw (DC = 8 + your Consitution modifier + your proficiency bonus) or take 1d8 cold damage for every turn it is in contact with you. 
\subparagraph{Spine Shards} The icy shards which grow from your spine erupt off of your body. When you take the Attack action, you can replace one of your attacks with your Spine Shards. All creatures in a 15 foot cone must make a Dexterity saving throw (DC = 8 + your Consitution modifier + your proficiency bonus), taking 1d10 cold damage on a failed save, or half as much on a successful save. This damage increases by 1d10 when you reach 5th level (2d10), 11th level (3d10), and 17th level (4d10) You can use this feature a number of times equal to your proficiency bonus, and regain use of it after a long rest.
\subparagraph{Languages}You can speak, read, and write Common and Crystallin.

\raggedbottom
\pagebreak

\section{Flameblooded}
\textit{As I stood at the edge of the clearing, I watched with a mix of awe and trepidation as a figure emerged from a cloud of smoke. The woman's skin glowed a fiery orange, and her eyes eyes were fixed upon me with a predatory intensity, as she twirled two gleaming daggers in he hands. The flameblooded moved with a grace and fluidity that I had never seen before, as if the daggers were simply an extension of their own body.\\
I knew my belongings would not be mine for much longer, but still I could not help be captivated by the glow of the woman, as she approached, as fluid as smoke. Her confidence, her assertiveness, made me believe that I could have some of mine own.} \\ \begin{flushright}---Malmalor Lumena \end{flushright}
\subsection{Iona's Children}
 Iona, the goddess of fire and destruction, was believed to have infused her own divine essence into the very veins of her Chosen Race. Legend has it that Iona created the flameblooded to be her loyal followers and warriors, imbuing them with the power of fire and passion to serve as her champions.\\
Flameblooded are characterized by their distinct physical features, including yellow to orange skin, bright fiery eyes, and hair that shimmers like flames in the sunlight. Their veins run with a primordial fire, giving them a natural affinity for flames, and the ability to control and manipulate them to their will. They are also known for their incredible endurance and resilience to heat, which allow them to thrive in the harsh desert environment they call home.\\
The flameblooded have a rich cultural history, centered around their devotion to Iona and their belief in the power of fire to cleanse and purify. They are deeply connected to the land they inhabit, and view the desert as a place of spiritual renewal and transformation. They are fierce warriors and skilled craftspeople, known for their intricate metalworking and jewelry-making skills, which often feature fiery motifs and designs inspired by their connection to the flame. Despite their intense devotion to Iona and the power of fire, flameblooded are also known for their passionate and tempestuous natures.
\subsection{A Charred Soul}
Coal holds a special significance in the rituals of the flameblooded. As a race that is inherently connected to the power of fire, coal is seen as a symbol of transformation and purification. During important ceremonies and rituals, flameblooded will often use coal to create large fires of purification. The heat and light of the flames are believed to awaken the innate power of the flameblooded, allowing them to connect more deeply with Iona.\\
In addition to its symbolic importance, coal is also used practically in many aspects of flameblooded life. The harsh desert environment in which they live often makes it difficult to find reliable sources of fuel for cooking and heating, and coal provides a steady and long-lasting source of heat. It is also used in the creation of their intricate metalworking, which often features fiery motifs and designs. For the flameblooded, coal is both a practical necessity and a powerful symbol of their connection to the divine power of fire.
\subsection{Life on the Road}
The flameblooded, despite settling in cities, have a tendency to lead a life of wandering. Their wanderlust nature often leads them to reject the confines of settled society in favor of a more nomadic existence. However, their penchant for independence often leads them astray, and many of them are known to succumb to a life of piracy and other forms of roguish activity.\\
Their natural connection to fire and formidable strength makes the flameblooded a formidable force in combat, which often makes them attractive candidates for a life of adventure and danger. Their sense of loyalty and duty to their own people often binds them together, making them an unstoppable force that few would dare cross. This reputation has earned them admiration and respect, and many consider them to be living embodiments of the free spirit of fire.\\
However, this wandering lifestyle often puts them at odds with settled societies, and they may be seen as outsiders and troublemakers. The flameblooded's tendency towards banditry often puts them at odds with law enforcement and other powerful groups that seek to bring them to justice. Despite these risks, many flameblooded continue to embrace the life of a wanderer, seeing it as the only true path to freedom and independence.
\subsection{Flameblooded Traits}
\subparagraph{Ability Score Increase}Your Charisma score increases by 2 and your Dexterity score increases by 1.
\subparagraph{Age}Flameblooded age in a similar manner to humans, reaching physical maturity around 18, and then usually live from 75 to 100 years.
\subparagraph{Alignment}Flameblooded have hot tempers, and tend to chaotic alignments.
\subparagraph{Size}Flameblooded are of a similar size to humans, ranging from 5 to 6 feet tall, and 130 to 180 pounds. Your size is Medium. 
\subparagraph{Speed}Your base walking speed is 30 feet. 
\subparagraph{Darkvision} You can see in dim light within 60 feet of you as if it were bright light, and in darkness as if it were dim light. You can't discern color in darkness, only shades of gray.
\subparagraph{Blood of Fire} You have resistance to fire damage. Any creature that grapples you or restrains you with its body must make a Constitution saving throw (DC = 8 + your Consitution modifier + your proficiency bonus) or take 1d8 fire damage for every turn it is in contact with you. 
\subparagraph{Scalding Sword} Your melee weapons which are made of metal deal an extra 1d4 fire damage. You cannot gain this benefit if the weapon has wooden parts or if the handle is not made of metal.
\subparagraph{Languages}You can speak, read, and write Common and Ignan.
\section{Lurkers}
\section{Seafolk}
