\chapter{New Playable Class and Subclasses}\label{Mentalist}
\section{Mentalist}

An arborean bends down, inspecting the muddy ground. The subtle outline of imprints is all that he needs to pursue the poachers who set flame to his forest. A human woman reaches her arm out, just in time to stop the king from taking a sip. The faintest difference in hue is all it takes for this observant servant to spot the poison in the drink. An aged tiefling man lashes out with his cane, striking the orc in the weak spot between the joints of her armor that only he noticed. He predicts her retaliation, and despite his elderly body, effortlessly dodges a devastating blow.

\subsection{Detectives and Investigators}

Every great Rogue Thief will eventually draw the attention of an equally great detective, determined to track them down. Some do it for the sense of justice, but many do it purely for the thrill of solving a great mystery. Mentalists, known for their exceptional cleverness, perceive the world as inherently predictable. They employ their brilliant logical skills to effectively track and pursue their targets.

Lords and sheriffs frequently extend substantial rewards for the services of a Mentalist, yet a Mentalist selectively chooses cases that captivate their interest. Financial support is seldom a concern for a Mentalist. If they prioritized wealth, their talents could be employed to discern the subtle facial cues of unsuspecting gamblers and amass substantial fortunes.

\subsection{Seeing What Others Miss}

Mentalists inherently observe their surroundings with acute awareness. Occasionally, this heightened sensitivity becomes overwhelming as their senses absorb more than the average person's. Identifying traps and ambushes pose little challenge for a Mentalist. Even during sleep, a Mentalist's body remains vigilant, discerning the slightest changes in the sounds of nearby wildlife.

In combat, many Mentalists might seem physically outmatched, but they are rarely outwitted. They possess the ability to discern a monster's vulnerabilities, enabling a well-aimed strike with a mere walking stick or cane to inflict severe damage. A Mentalist can anticipate and counter a spellcaster's attacks or glean a target's deepest fears from their body language.

A Mentalist will never let the same enemy get the best of them twice. If an enemy escapes or bests them once, a Mentalist will replay the moment in their mind until they have determined how to prevent the same from happening again.

\subsection{Creating a Mentalist}

Mentalists are not crafted; instead, they are born with a preternatural gift, allowing them to perceive the world as it truly is. While inherently brilliant, Mentalists frequently grapple with a certain degree of antisocial tendencies. Their awareness of the flaws and secrets of those they encounter often taints their perspective of people and society with a negative hue.

Perhaps your character traverses the world in search of anything intriguing to alleviate the existential boredom of loneliness. Alternatively, they may be on a quest to find the one criminal who has successfully outwitted them. Is your Mentalist harboring anger toward the world, or are they fascinated by its intricacies? Do they revel in showcasing their talents, or do they prefer to keep their abilities private? Can your Mentalist foster close relationships with others when every secret flaw is laid bare to their heightened perception?

 \subsubsection{Quick Build}
 You can make a Mentalist quickly by following these suggestions. First, Intelligence should be your highest score, followed by Dexterity, then Charisma. Second, if you are not choosing the Branded background in the Kiraki Campaign setting, choose the Folk Hero or Sage background.
 \section{Class Features}
 As a Mentalist, you gain the following class features.
\subsubsection{Hit Points}
\noindent\textbf{Hit Dice:} 1d8 per Mentalist level\\
\noindent\textbf{Hit Points at 1st Level:} 8 + your Constitution modifier\\
\noindent\textbf{Hit Points at Higher Levels:} 1d8 (or 5) + your Constitution modifier per Mentalist level after 1st

\subsubsection{Proficiencies}
\noindent\textbf{Armor:} None\\
\noindent\textbf{Weapons:} Canes, walking sticks, or other similarly styled Mentalist Weapons\\
\noindent\textbf{Tools:} Thieves' Tools or Forgery Kit\\
\noindent\textbf{Saving Throws:} Intelligence, Charisma\\
\noindent\textbf{Skills:} Choose four from Perception, Investigation, History, Insight, Persuasion, Deception, and Intimidation

\subsubsection{Equipment}
You start with the following equipment, in addition to the equipment granted by your background:
\begin{itemize}
  \item A cane, walking stick, or pointer stick, which serves as your Mentalist Weapon for striking enemies in their most vulnerable locations.
  \item (a) any simple weapon, or (b) 2 daggers
  \item (a) an adventurer's pack, or (b) a burglar’s pack
\end{itemize}


\subsubsection{Stick Fighting}
At 1st level, your intuition and meticulous study of anatomy allow you to strike at creatures in their most vulnerable spots using a simple stick or cane. The cane is considered a light, simple Mentalist Weapon, dealing non-magical bludgeoning damage. You can only wield one Mentalist Weapon. The damage of the Mentalist Weapon is determined by the character's Mentalist level, as shown in the Mentalist table.

While wielding only your Mentalist Weapon (without a shield, other weapons, or focuses), you gain the following benefits:
\begin{itemize}
    \item You can add your Intelligence modifier as well as your Dexterity modifier to both your attack and damage rolls.
    \item For creatures with no levels in Mentalist, a Mentalist Weapon deals a flat 1d4 damage.
\end{itemize}
\subsubsection{Unarmored Defense}
At 1st level, when you are not wearing any armor and not wielding a shield, your Armor Class equals 10 + your Dexterity modifier + your Intelligence modifier.
\subsubsection{Keen Eye} 
At 2nd level, your accelerated information processing grants you advantage on Perception, Investigation, and Insight checks.
\subsubsection{Anticipation} 
At 2nd level, you acquire the ability to take the Dodge action as a bonus action a number of times equal to your proficiency bonus. You regain all uses of this feature after a short or long rest.
\subsubsection{Logical Mind} 
At 4th level, you become immune to the Charmed condition.

At 9th level, you become immune to being Frightened.
\begin{DndTable}[header=The Mentalist, width=0.59\textwidth]{m{0.04\textwidth}m{0.07\textwidth}m{0.06\textwidth}m{0.2\textwidth}}
    \textbf{Level}  & \textbf{Proficiency Bonus} & \textbf{Mentalist Weapon Damage} & \textbf{Features}\\
    1st  & +2 & 1d4 & Unarmored Defense, Stick Fighting\\
    2nd & +2 & 1d4 & Keen Eye, Anticipation\\
    3rd & +2 & 1d4 & Mentalist Specialty \\
    4th & +2 & 1d6 & Ability Score Improvement, Logical Mind\\
    5th & +3 & 1d6 & Extra Attack, Deduction \\
    6th & +3 & 1d6 & Mentalist Specialty Feature\\
    7th & +3 & 1d6 & Observist, Watchful Eyes\\
    8th & +3 & 1d6 & Ability Score Improvement \\
    9th & +4 & 2d4 & Logical Mind Improvement, Unavoidable presence \\
    10th & +4 & 2d4 & Precise Attack \\
    11th & +4 & 2d4 & Mentalist Specialty Feature \\
    12th & +4 & 2d4 & Extra Attack \\
    13th & +5 & 2d4 & Act Natural \\
    14th & +5 & 2d4 & Watchful Eyes Improvement \\
    15th & +5 & 2d4 & Let's Try That Again \\
    16th & +5 & 2d4 & Ability Score Improvement \\
    17th & +6 & 1d4+1d6 & Mentalist Specialty Feature \\
    18th & +6 & 1d4+1d6 & True Understanding \\
    19th & +6 & 1d4+1d6 & Ability Score Improvement \\
    20th & +6 & 1d4+1d6 & Predictive Master \\
\end{DndTable}  
\subsubsection{Extra Attack}
At 5th level, you gain the ability to make a second strike with your Mentalist Weapon as part of your attack action.

At 12th level, this capability increases, allowing you to make a third strike with your Mentalist Weapon as part of your attack action.
\subsubsection{Deduction}
At 5th level, you can utilize your bonus action to attempt to discern all of a creature's resistances, immunities, and susceptibilities. Make an insight check with a DC equal to the creature’s (CR/level + 10) to have this information revealed to you.
\subsubsection{Observist}
At 7th level, your mastery of observation and reflection grants you a significant advantage. If you have previously fought an individual or an identical enemy, you gain advantage on all attacks against that being, provided you have taken the time to reflect during a short or long rest in between encounters. Creatures are considered identical if they belong to the same species and have an Intelligence score lower than 10.
\subsubsection{Watchful Eyes}
At 7th level, your passive perception cannot be lower than 18. Additionally, you automatically detect any traps with a DC lower than 16.

At 14th level, your passive perception cannot be lower than 20, irrespective of your Wisdom modifier. When you roll to detect any traps, treat any dice roll at or below a 9 as a 10. In addition, you cannot be surprised.
\subsubsection{Unavoidable Presence}
At 9th level, the reach of your Mentalist Weapon extends to 10 feet. Additionally, you gain the flanking bonus if any ally is within 10 feet of the target, even if positioning would not typically grant flanking. Furthermore, your Mentalist Weapon attacks are now considered magical.
\subsubsection{Precise Attack}
At 10th level, when you score a critical hit, both the dice rolls and the flat damage modifiers are doubled.
\subsubsection{Act Natural}
At 13th level, you can use a bonus action to take the Hide action.

Outside of combat, you can seamlessly blend into the environment, making it appear as if you are a natural part of the surroundings to an enemy you have not directly engaged. A creature must have a passive Perception higher than 20 or actively be on the lookout, making a successful DC 17 Wisdom (Perception) check to become aware of your presence.
\subsubsection{Let's Try That Again}
At 15th level, you gain the ability to prefigure conversations before they occur. If an non-combat interaction with a creature turns unfavorable, you can designate the last minute of this conversation or interaction as a prediction and then resume the conversation or interaction from one minute earlier. You can use this feature once and regain its use after a short or long rest.
\subsubsection{True Understanding}
At 18th level, your understanding of what makes a creature tick is so profound that you can dismantle its identity with just a few words. You can take an action to compel a target within hearing range of you to make a Wisdom saving throw (DC 10 + your Charisma modifier + your Intelligence modifier). The creature takes 80 psychic damage on a failed save, or half as much on a successful one. To use this feature, you must have observed the creature for more than one minute, the target's Intelligence score must be greater than 6, and the target must be able to hear and understand you. You regain the use of this feature after a long rest.
\subsubsection{Predictive Master}
At 20th level, you can utilize your reaction at the start of a round of combat to foresee the precise unfolding of the entire round. You influence the flow, causing all enemies to miss all attacks directed against you or allies within 10 feet of you for the entire round. Additionally, you effortlessly pass all saving throws during the round. You regain the use of this feature after a short rest.
\section{Mentalist Observation Specialties}
At 3rd level, you learn to specialize in one type of observation and this gains you features at 3rd level and again at 6th, 11th, and 17th level. 
\subsection{Observer of Body}
You've dedicated yourself to the prediction and anticipation of attacks, investing years in studying the physiology of both humanoid and monstrous adversaries. Your expertise enhances your ability to evade and strike against non-magical creatures.
\subsubsection{Can't Touch This}
At 3rd level, you acquire the skill to roll with advantage against any effects stemming from a non-magical attack or contact by a creature that necessitates Strength or Dexterity-based saving throws or contested ability checks. Additionally, you gain proficiency in the Acrobatics and Athletics skills.
\subsubsection{Find the Weak Point}
At 6th level, you develop an understanding of creatures' pressure points, allowing you to inflict maximum damage. The dice roll you need to score a critical hit is reduced by 1, and you can bypass an enemy's resistance to your damage types.
\subsubsection{Unarmored Defense Improvement}
At 11th level your unarmored AC increases by 2.
\subsubsection{Stunning Smack}
At 17th level, upon the first successful hit of the round with your Mentalist Weapon, the targeted creature must succeed on a Constitution saving throw (DC 6 + your Intelligence modifier + your Dexterity modifier) or be stunned until the end of its next turn.
\subsection{Observer of Mind}
You have specialized in the mastery of your opponents' minds. The thoughts and fears of those around you become yours to manipulate, as the slightest micro-expressions they make reveal their weaknesses to you. Your mind becomes a fortress to others, ensuring that you never give away any of the details you use to assess them.
\subsubsection{Extra Proficiencies}
At 3rd level, you acquire the adept skill to deftly influence creatures. You may utilize twice the standard proficiency modifier for Deception, Persuasion, and Insight checks if you are already proficient in them, or you gain proficiency if you are not. Additionally, you gain proficiency in disguise kits.
\subsubsection{Words of Reckoning}
Also, at 3rd level, as a bonus action, you can scrutinize any creature. As long as it possesses an Intelligence score greater than 9 and can understand the language you speak, you can utter a sentence that inflicts pain. The target must make a Wisdom saving throw (DC 8 + your proficiency bonus + your Intelligence modifier). On a failed save, the target takes 1d4 psychic damage. The damage increases by 1d4 at 5th, 11th, and 17th levels. You can use this feature three times, and you regain the use of this feature after completing a short or long rest.
\subsubsection{Mind Fortress} 
At 6th level, you gain the ability to cast the Charm Person spell at 2nd level, with Intelligence serving as your spellcasting modifier. You can use this feature to cast the spell as an action once. You regain the use of this feature after completing a short or long rest.

Furthermore, you attain immunity to psychic damage, and you gain advantage on Intelligence, Wisdom, and Charisma saving throws.
\subsubsection{Frightful Presence}
At 11th level, your sociopathic inclinations and piercing gaze enable you to instill fear in any creature within 30 feet that can see you. As an action, you can compel all creatures in this range to make a Wisdom saving throw (DC 8 + your proficiency bonus + your Intelligence modifier) or become Frightened of you for 1 minute. A creature can repeat the saving throw at the end of each of its turns, ending the effect on itself upon a success. If a creature's saving throw is successful or the effect ends for it, the creature becomes immune to your Frightful Presence for the next 24 hours.
\subsubsection{Penetrating Gaze}
At 17th level, you attain the capability to effortlessly delve into the thoughts of others. As an action, you can, at will, discern the surface thoughts of any creature with an Intelligence score between 6 and 18, provided it is fully visible to you, unless a spell or effect specifically obstructs such insight..
\subsection{Observer of Magic}
You have honed your expertise in foreseeing and preempting spellcasting. Your keen observations and intellectual prowess allow you to disrupt and mitigate the impact of spellcasters. Infamous mage bandits tremble at the prospect of your inexorable approach to their very doorsteps.
\subsubsection{Wisely Dodged}
At 3rd level, you acquire the capability to roll with advantage when making saving throws against spells or magical effects that necessitate a Wisdom or Intelligence saving throw. Additionally, you gain proficiency with an alchemist’s kit and in Arcana checks.
\subsubsection{Anti-Mage}
At 6th level, you learn the spell Counterspell, which you can cast at a spell level equivalent to half your Mentalist level, rounded down to a maximum of 7th level. Your spellcasting modifier for this feature is Intelligence. You can employ this ability a number of times equal to your proficiency bonus, and you regain all expended uses after completing a long rest.
\subsubsection{Spell Thief}
At 11th level, you can utilize a bonus action to gain profound insight into a creature’s magical prowess. Upon successfully passing an Arcana check (DC equal to the creature's CR/level + 8), you promptly acquire knowledge of the creature’s complete spellcasting repertoire. Against spells you are aware are within their capabilities, you have advantage on saving throws and experience no damage or adverse effects upon succeeding in a saving throw.
\subsubsection{Magi's Demise}
At 17th level, you effortlessly succeed on saving throws against magical effects with a DC lower than 18. Furthermore, you develop resistance to force, fire, and radiant damage from spells.
\subsection{Observer of Nature}
You have specialized in study of the natural world. You are particularly adept at tracking and identifying natural phenomena. You have used your talents to become close with nature.
\subsubsection{Touch Grass}
At 3rd level, you acquire proficiency in the Nature and Survival skills. Additionally, you have the option to select two cantrips from the Druid or Ranger spell lists, utilizing Intelligence as your spellcasting modifier for these chosen cantrips.
\subsubsection{Elemental Understanding}
At 6th level, you attain resistance to a damage type of your choice: Thunder, Lightning, Cold, Fire, or Acid. At 11th level, you can select an additional damage type to gain resistance to, and once more at 17th level.
\subsubsection{Natural Detective}
At 11th level, you gain the ability to glean insights from your surroundings at will. By dedicating 10 minutes to this process, you can extract information from the environment as if you had cast the Commune with Nature, Speak with Plants, Speak with Animals, or Speak with Dead spells, all without employing any magic but with a range constrained by your sensory perception. While there is no physical reply, you obtain a natural understanding of events that unfolded in the designated area.
\subsubsection{Weatherman}
At 17th level, your mastery over predicting natural phenomena evolves into a prescriptive capability. You can invoke the Control Weather spell, but with a casting time reduced to 1 action. You can employ this feature once, and regain its use after completing a long rest.
\raggedbottom
\pagebreak

\section{College of Colors Bard}\label{Colors}
Most bards entertain with their voice, charming an audience with beautiful song and performance. A College of Colors bard instead works with visual mediums to entertain and express. They are masterful illustrators, and carry around canvases and paints with them to create works of art on the spot while performing. Their performances are more than just songs and stories --- they are visual spectacles that leave audiences in awe.\\
These bards use their art to weave spells and create illusions, painting scenes that come to life before the eyes of their audience. They can create illusions that are so vivid, they can even manipulate reality, using their art to reshape the world around them. They have a deep understanding of color and form, and can use their art to evoke powerful emotions in their audience, whether it be joy, sadness, or fear. They are free spirits, traveling from place to place, seeking inspiration and new experiences.

\subsection{Paint-Marked}
When you join the College of Colors at 3rd level, you choose to use a paintbrush rather than a musical instrument as your spellcasting focus. You gain proficiency with \textbf{Painter's Supplies}.\\
In addition, you gain the ability to make your weapon attacks infused with a magical paint. When you make a successful weapon attack against an enemy, you can choose to spend one of your \textbf{Bardic Inspiration Dice} to mark the target with paint. The paint mark lasts one round. The color of paint they are marked with determines the effect:
\begin{itemize}
    \item \textit{Red:} The target is marked to bleed. Every attack the target is hit with which does any piercing, slashing, or bludgeoning damage, deals an extra amount of damage equal to your proficiency bonus.
    \item \textit{Yellow:} The target is marked to cower. The target becomes frightened of you.
    \item \textit{Blue:} The target is marked for despondence. The target loses motivation and has disadvantage on all saving throws.
\end{itemize}
At 14th level, you gain the ability to mix your paints. When you use your Paint-Marked feature, you can make orange paint which mixes red and yellow paints and gets both effects, or green paint which mixes blue and yellow, or purple paint which mixes red and blue.
\subsection{Caricature}
At 3rd level, as an action, you can craft a quick sketch which comes to life, as a caricature of an enemy you can see. It is created in an open space within 15 feet of you. The caricature counts as an illusion, but can be targeted by whatever effects can target the original creature. While the caricature is alive, and within 60 feet of the creature that inspired the caricature, it siphons a number of hit points off of the original, equal to your proficiency bonus at the end of each of the target's turns. The caricature has an AC of 10, and cannot move. The caricature starts with 10 hit points, and can gain temporary hitpoints up to three times the caster's bard level, by siphoning them off of the original. You can use this feature twice, and regain all uses after a long rest. You can only make one active caricature of any given creature.
\subsection{Portrait Walking}
At 6th level, you gain the ability to enter paintings as a bonus action. A replica of yourself in the style of the painting appears in the image. While you are in a painting, you can not be targeted. If the painting is destroyed while you are in it, you reemerge outside of the painting and suffer 5d6 force damage. \\
When you are inside the painting, the only actions you can take are leaving the painting, or traveling. You can travel to any other painting that you are familiar with which was made by the same artist, as long as it is on the same plane of existence as the painting you entered. \\
At 9th level, you can bring one other person into the painting to travel with you. You can only bring another person once per day. At 14th level, you can bring up to three people with you, or the same person on three separate travels.
\subsection{Drain Color}
At 14th level, you gain the ability to summon colors to your paintbrush. As a reaction, you can choose to summon all of the damage of one type dealt on an attack that hits an allied creature within 15 feet of you to be harmlessly absorbed in your paintbrush. This absorbs the damage from all targets if the damage effect multiple creatures. You can do this if the damage color is red (fire damage), orange (necrotic damage), yellow (lightning damage), green (acid~damage), blue (cold damage), or purple (poison damage). You can do this a number of times equal to your Charisma modifier, and regain all uses after a long rest.
\raggedbottom
\pagebreak
\section{Circle of Rainbows Druid}\label{Rainbow}
The Circle of Rainbows is a rare and enigmatic druidic order that reveres the elusive rainbow, and channels the power of color and light in their magic. They are a joyful and playful group, often seen dancing and frolicking in fields of wildflowers, surrounded by a shimmering aura of rainbow light.\\
These druids have a deep connection to the natural world, and use their magic to protect and preserve it. They can call forth rainbows in even the most inhospitable of environments, so vivid they seem to be made of pure magic, and use them to heal and energize those around them.


\subsection{Circle Spells}
Your link to rainbows and light grants you access to certain spells. At 2nd level, you learn the Hand of Radiance cantrip.\\
At 3rd, 5th, 7th, 9th, and 17th level you gain access to the spells listed for that level in the Circle of Rainbows Spells table. Once you gain access to one of these spells, you always have it prepared, and it doesn't count against the number of spells you can prepare each day. If you gain access to a spell that doesn't appear on the druid spell list, the spell is nonetheless a druid spell for you.
\begin{DndTable}[header=Circle of Rainbows Spells, width=0.45\textwidth]{XX}
    \textbf{Druid Level}  & \textbf{Circle Spells} \\
    2nd & Hand of Radiance\\
    3rd & Color Spray, Skywrite\\
    5th & Hypnotic Pattern, Daylight\\
    7th & Storm Sphere, Color Wheel\\
    9th & Wall of Light, Commune with Nature \\
   17th & Prismatic Spray, Prismatic Wall
\end{DndTable}  
\subsection{Pot of Gold}
At 2nd level, you can see places touched by rainbows. Once per day, as long as you are in a natural outdoor environment where rainbows could occur, you can identify a location touched by a rainbow, and find a treasure that was buried at the rainbow's end. By digging a 5 foot deep hole at that location, you can unearth a pot of treasure, with the amount of treasure determined by rolling a d100 and comparing it to the Pot of Gold Rewards Table below.
\begin{DndTable}[header=Pot of Gold Rewards, width=0.45\textwidth]{XX}
    \textbf{Dice Roll}  & \textbf{Reward} \\
    1-40 & 5 Gold\\
    41-80 & 10 Gold\\
    81-90 & 50 Gold\\
    91-99 & 100 Gold\\
    100 & 1000 Gold
\end{DndTable}  
\subsection{Rainbow Road}
At 2nd level, as an action, you can expend a use of your wildshape, to create a healing rainbow aura behind you as you move. Your entire path of movement for the round is traced out by a rainbow aura upon the ground. Any creatures of your choice who pass through the rainbow aura can regain a number of hit-points equal to your proficiency bonus once per round. Your rainbow aura from your most recent movement lasts until the start of your next turn. Your rainbow does not heal yourself even if you double back upon your path. Your Rainbow Road ability lasts for 1 minute.
\subsection{Chasing Rainbows}
At 6th level, you can become as elusive as a rainbow. If an enemy you can sense the location of moves toward you, you can use your reaction to simultaneously move an equal distance away from them in the same direction, so long as there is open space in the direction of motion.
\subsection{Color Blessing}
At 10th level, as a bonus action, you can choose to prioritize one color in your Rainbow Road aura. On subsequent turns you can choose to change the color as a bonus action if your Rainbow Road is still active. When an ally passes through your Rainbow Road aura, they are granted an additional bonus to the healing depending on the color chosen. The bonus lasts for the duration of the Rainbow Road aura, or until the receiver of the bonus passes through a rainbow road of a different color.
\begin{itemize}
    \item \textit{Red:} Advantage on Constitution Saving Throws
    \item \textit{Orange:} Advantage on Strength Saving Throws
    \item \textit{Yellow:} Advantage on Dexterity Saving Throws
    \item \textit{Green:} Advantage on Wisdom Saving Throws
    \item \textit{Blue:} Advantage on Intelligence Saving Throws
    \item \textit{Indigo:} Advantage on Charisma Saving Throws
    \item \textit{Violet:} +15 Speed
\end{itemize}
\subsection{Color Body}
At 14th level, you glow with radiance. After each long rest, you can choose a color. The color you choose grants you resistance to a type of damage until you next choose to change your color. The options are as follows:
\begin{itemize}
    \item \textit{Red:} Fire
    \item \textit{Orange:} Thunder
    \item \textit{Yellow:} Lightning
    \item \textit{Green:} Acid
    \item \textit{Blue:} Cold
    \item \textit{Indigo:} Radiant
    \item \textit{Violet:} Poison
\end{itemize}
\section{Punk Rogue}\label{Punk}
The Punk Rogue embodies a spirit of rebellion, driven by an innate desire to dismantle unjust structures and humble those who claim superiority. Their rage against authority manifests through powerful strikes toward figures in power. The Punk Rogue also channels this rage into an art form, turning their expression into a weapon against the system. The Punk Rogue's aptitude for sabotage allows them to leave a fiery mark on structures, and at the pinnacle of their rebellion, their impassioned cry disrupts even the cruelest oppressor, showcasing their enduring spirit in the quest for dismantling systems.
\subsection{Upstart}
At 3rd level, you can channel your rage at authority figures such as government officials, law enforcement, boss monsters, or similar, and deal an extra d6 of damage on your sneak attacks against them.
\subsection{Channel Your Rage}
Also at 3rd level, you can Channel Your Rage into an artform of your choice. You gain proficiency with a set of Artisan's Tools. or a musical instrument. You also gain proficiency in the Performance, and Intimidation skills.
\subsection{Rallying Cry}
At 9th level, as an action, you can rally your allies within 30 feet, and all of their successful melee attacks will deal an additional 1d8 Force damage for 1 round. You can use this feature twice, and regain its use after a long rest.
\subsection{Burn it Down}
At 13th level, on any hit against vehicles and structures, you can add double your normal amount of sneak attack die to the damage, and you can choose to make the sneak attack die deal fire damage. 
\subsection{Glorious Cause}
At 17th level, your impassioned fury against the system is potent enough to unsettle the convictions of even the most loyal of henchmen. As an action, you can vociferate at a creature within 30 feet that is a minion, weaker ally, or summon of an enemy, and can hear and comprehend you. The target must make a Charisma saving throw (DC equal to 8 + your proficiency bonus + your Charisma modifier). On a failure, the target is Charmed by you for 1 hour, and will rebel against their former leader and allies, joining you if you attack them. The target can remake the saving throw every time it takes damage. You can use this feature once, and regain use of it after a long rest.
\section{Way of Balance Monk}\label{Balance}
Monks of the Way of Balance seek to maintain the harmony between good and evil, and life and death. They wander the world in search of places where the balance is threatened, using their skills to heal and protect. Their powers originate from the chaos of imbalance, their ki yearns to rebalance this chaos and return the world to neutrality.\\
While a Way of Balance Monk can be of any alignment, good aligned Way of Balance Monks usually only appear in times of great evil and danger, and evil aligned ones only appear in times of great peace and harmony. Most will tend towards a more neutral alignment.

\subsection{Balanced Vitality}
At 3rd level, you acquire the ability to swap your life essence with another being. As a bonus action, you can expend a ki point when touching a friendly creature. If your current hit points exceed those of the creature, you exchange your life force with them. Both your hit points and the creature's are adjusted to the average of your previous current hit points. If this results in the creature having more hit points than their normal maximum, they can gain a number of temporary hit points, up to a maximum of 5.

If the friendly creature possesses more hit points than you, they have the option to permit you to siphon their life essence in the same manner.
\subsection{Balanced Abilities}
At 6th level, you acquire the ability to harmonize your abilities with a target. As an action, you can expend 2 ki points to endeavor to average one of your ability scores with a creature you can see within 30 feet. Conduct a contested skill check using the stat you intend to modify against a saving throw of the same stat for the creature. If you roll higher, your and the target's ability scores for that stat are averaged (rounding up on a half integer) for one minute.
\subsection{The Balance of Life and Death}
At 11th level, whenever you roll a death saving failure, you also accumulate one success. This rule doesn't apply if you incur a failure due to damage. If you simultaneously reach three failures and three successes, your character succumbs and dies.
\subsection{Mirror of Good and Evil}
At 17th level, your ki yearns to restore balance to the world. As an action, you can expend 4 ki points to attempt an instantaneous swap of the locations of two creatures you are familiar with. The creatures must be within 1 mile of you and each other, and they must be of opposite alignment on either the good and evil spectrum, or the lawful and chaotic spectrum. Creatures have the option to make a Wisdom saving throw against your ki save DC to resist the effect. Both creatures must fail their saving throws for the ability to take effect.

This ability is thwarted by spells or area effects that prevent teleportation, scrying, or similar effects on either creature or their locations. Such effects either prevent this ability from working entirely or necessitate additional saves or skill checks as dictated by those effects. In such cases, you, rather than the targets, make any required saves or skill checks.